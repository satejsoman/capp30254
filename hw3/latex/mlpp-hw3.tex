\documentclass[11pt]{article}

% packages
\usepackage{enumerate}
\usepackage{fancyhdr}
\usepackage{extramarks}
\usepackage{amsmath}
\usepackage{amsthm}
\usepackage{amssymb}
%\usepackage{amsfonts}
\usepackage{tikz}
\usepackage[plain]{algorithm}
\usepackage{algpseudocode}
\usepackage{lastpage}
\usepackage{units}
\usepackage[margin=0.75in]{geometry}
\usepackage{pgfplots}
\pgfplotsset{compat=1.16}
\usepackage{sectsty}
\usepackage{hyperref}
\usepackage{multicol}
\usepackage{mathtools}
\usepackage{tikz}
\usetikzlibrary{matrix}
\usepackage{listings}
\usepackage[plain]{algorithm}
\usepackage{algpseudocode}
\usepackage{mdframed}
\usepackage{booktabs}

\definecolor{sblue}{HTML}{5292c0}
\definecolor{sgreen}{HTML}{93c47d}
\definecolor{sorange}{HTML}{e69138}
\definecolor{codeblue}{rgb}{0.29296875, 0.51953125, 0.68359375}
\definecolor{codegreen}{rgb}{0.47265625, 0.62890625, 0.40234375}
\definecolor{codegray}{rgb}{0.95703125, 0.95703125, 0.95703125}
\definecolor{codecrimson}{rgb}{0.87109375,0.3984375,0.3984375}

\lstset{frame=tb,
  backgroundcolor=\color{codegray},
  aboveskip=3mm,
  belowskip=3mm,
  showstringspaces=false,
  columns=flexible,
  basicstyle={\small\ttfamily},
  numbers=left,
  numberstyle=\tiny\color{gray},
  keywordstyle=\color{codeblue},
  commentstyle=\color{codegreen},
  stringstyle=\color{codecrimson},
  breaklines=true,
  breakatwhitespace=true,
  tabsize=4,
  frame=tlbr,framesep=4pt,framerule=0pt,
  literate={`}{\`}1,
}

% colors
\definecolor{sblue}{HTML}{5292c0}
\definecolor{sgreen}{HTML}{93c47d}
\definecolor{sorange}{HTML}{e69138}

% sectioning magic
\counterwithin*{equation}{section}
%\numberwithin{equation}{section}

% fonts
\usepackage{fontspec}
\newfontfamily\headerfontlt{ITC Franklin Gothic Std Book}
\newfontfamily\headerfont{ITC Franklin Gothic Std Demi}
\usepackage[urw-garamond]{mathdesign}
\usepackage{garamondx}
\usepackage[italic]{mathastext}

\newcommand{\printsection}[1]{\normalfont\headerfontlt{{{#1}}}}
\newcommand{\printsubsection}[1]{\normalfont\headerfontlt\textcolor{darkgray}{{#1}}}
\newcommand{\printsubsubsection}[1]{\normalfont\headerfontlt{{#1}}}
\allsectionsfont{\printsection}
\subsectionfont{\printsection}
\subsectionfont{\printsubsection}
\subsubsectionfont{\printsubsubsection}


\renewcommand{\headrulewidth}{0.1pt}
\renewcommand{\headrule}{\hbox to\headwidth{\color{gray}\leaders\hrule height \headrulewidth\hfill}}
\renewcommand{\footrulewidth}{0.0pt}

\newcommand{\op}[1]{\textrm{\small\printsubsection{\MakeUppercase{#1}}}\,}
\newcommand{\opns}[1]{\textrm{\small\printsubsection{\MakeUppercase{#1}}}}
\newcommand{\Partial}[1]{\partial\hspace{-0.2ex}{#1}}
\newcommand{\D}[1]{\mathrm{d}#1}
\newcommand{\E}[1]{\mathbb{E}{\left[\,#1\,\right]}}
\newcommand{\var}[1]{\mathrm{var}\left( #1 \right)}
\newcommand{\cov}[1]{\mathrm{cov}\left( #1 \right)}
\newcommand{\pr}[1]{\mathrm{Pr}\left( #1 \right)}
\newcommand{\given}{\,|\,}
\renewcommand{\det}{\mathrm{det\,}}
\renewcommand{\det}[1]{\op{det}\left(#1\right)}
\renewcommand{\ker}{\mathrm{Ker\,}}
\newcommand{\trace}[1]{\op{trace}\left(#1\right)}
\newcommand{\nul}[1]{\op{nul}\left(#1\right)}
\newcommand{\col}[1]{\op{col}\left(#1\right)}
\newcommand{\row}[1]{\op{row}\left(#1\right)}
\newcommand{\rank}[1]{\op{rank}\left(#1\right)}
\renewcommand{\dim}[1]{\op{dim}\left(#1\right)}
\newcommand{\im}{\op{Im\,}}
\newcommand{\reals}{\mathbb{R}}
\newcommand{\complex}{\mathbb{C}}
\renewcommand{\vec}[1]{\mathbf{#1}}
\newcommand*{\matr}[1]{\mathbfit{#1}}
\newcommand*{\tran}{^{\mkern-1.5mu\mathsf{T}}}
\newcommand*{\conj}[1]{\overline{#1}}
\newcommand*{\hermconj}{^{\mathsf{H}}}
\newcommand*{\inv}{^{-1}}

%\setlength\parindent{0pt}
\pagecolor{gray!1}


% header/footer
\topmargin=-0.65in
\pagestyle{fancy}\fancyhf{} 
\lhead{\headerfontlt{\textcolor{darkgray}{Satej Soman \textcolor{gray}{$\big/$} CAPP30254, Spring 19 \textcolor{gray}{$\big/$}} Homework 3}}
\rhead{\headerfontlt{\textcolor{gray}{$\big[$} \thepage\ \textcolor{gray}{$\big/$} \pageref{LastPage} \textcolor{gray}{$\big]$}}}

\begin{document}
\begin{titlepage}
\raggedleft\huge\headerfontlt{
\textcolor{darkgray}{Satej Soman\\
CAPP30254: Machine Learning for Public Policy\\
Spring 2019}}

\vspace{240pt}
\Huge\headerfontlt{\textcolor{darkgray}{HW 3\\MACHINE LEARNING PIPELINE\\IMPROVEMENTS \& EVALUATION}}
\vfill
\normalfont \normalsize
\tableofcontents

\end{titlepage}
\section*{Notes}
\begin{itemize}
\item Representative code snippets are interspersed with analysis and explanations below; all code is available on GitHub: \url{https://github.com/satejsoman/capp30254/tree/master/hw3}.
\item The pipeline library lives in the \texttt{code/pipeline} directory, while the sample application which imports the library is \texttt{code/donors\_choose.py}.
\end{itemize}

\section{Coding: Pipeline Improvements}
\subsection{Improvements}
The following improvements have been made to the \texttt{pipeline} library:
\begin{itemize}
\item A stage to generate train/test data splits. Due to the object-oriented design, the current implementation can easily be overriden by subclassing or monkey-patching in specific applications.
\item Additional metrics, besides accuracy, have been added to the model evaluation. Specifically, precision, recall, and ROC-AUC have been added to the model evaluation stage.
\end{itemize}

\subsection{Additional Models}
In the original pipeline design, the model was not hard-coded, so pipeline runs can be parametrized by the model implementation: 

\begin{lstlisting}[language=Python,numbers=none]
donors_choose_preprocessors = [
    ...
]
donors_choose_feature_generators = [
    ...
]

models_to_run = { 
    ...
}

def model_parametrized_pipeline(description, model):
    return Pipeline(input_path, "funded_in_60_days",
        summarize=False,
        data_preprocessors=donors_choose_preprocessors,
        feature_generators=donors_choose_feature_generators,
        name="donors-choose-" + description,
        model=model
        output_root_dir="output")

for (description, model) in models_to_run.items():
    model_parametrized_pipeline(description, model).run()
\end{lstlisting}

The above code is purely representative; in implementation, the data generation and feature preprocessing are done in a separate pipeline. The results of this pipeline are serialized and fed in as the inputs to a pipeline that solely trains and tests models. In this way, computation to process data and create feature vectors is not repeated for every trial run.


\section{Analysis: Application to DonorsChoose Project Funding Viability}
\subsection{Data Exploration}

\subsection{Feature Selection}

\section{Report: Classifier Performance}

\end{document} 
